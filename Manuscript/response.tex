\documentclass[12pt]{letter}
\usepackage[a4paper,left=2.5cm, right=2.5cm, top=2.5cm, bottom=2.5cm]{geometry}
\usepackage{color}
\usepackage[usenames,dvipsnames]{xcolor}
\usepackage[osf]{mathpazo}
\usepackage{graphicx}
\usepackage{amsmath,amssymb}
\signature{Natalie Cooper \\ Gavin Thomas \\ Chris Venditti \\ Andrew Meade \\ Rob Freckleton}
\address{Department of Life Sciences \\ Natural History Museum \\ Cromwell Road \\ London, SW7 5BD \\ nhcooper12@gmail.com}
\longindentation=0pt
\begin{document}

\begin{letter}{}
\opening{Dear Dr O'Hara}

Please find enclosed our resubmission of our manuscript ``Shedding Light on the ``Dark Side'' of Phylogenetic Comparative Methods" (MEE-15-10-697). We include below details of how we have dealt with the editor's and reviewers' comments. The comments are in blue and our response is in black.

Thanks to Bob, Amy and Simon for all your helpful comments. This was a tricky paper to write, and we debated heavily about whether to write it at all. Finding the right tone was indeed very difficult, and we clearly haven't completely succeeded there. Also while we had started to "Do Something" about some of this around the time I first presented this, in the interim both Rich and I have moved countries and jobs (Rich out of academia) delaying some projects and derailing others completely!

\textcolor{blue}{\textbf{Bob O'Hara's comments}}\\

\textcolor{blue}{The reviewers and Associate Editor are positive about your work, however there are a few issues that need attention. Therefore, I invite you to respond to the reviewer(s)' comments and revise your manuscript. This might be closer to a Major Revisions in some ways: there are several issues to be addressed. At times the manuscript comes across as a bit whiny ("Why don't you lot do something?"). Tone can be difficult to get right, and there are some good suggestions (e.g. use Markdown to show your code), but after reading this I think I might avoid PCMs, as it seems there are a lot of pitfalls that I won't know about unless I work with someone who knows about them. Evidently it is not my destiny to come over to the Dark Side.}

We have tried to correct the tone where we can. 

\textcolor{blue}{I think the authors' suggestions that Someone Needs to Do Something would be stronger if they Started to Do Something. For example, they mention the BES QSIG's FGE pages, so why not start up a page on PCMs? An intro to running PCMs would be a necessary start, plus perhaps a Checklist of Things to be Checked (And How to Check Them). If the authors could find a day or two to set these up, then I think this would give the manuscript a more positive tone by announcing it, and might (if you are really really lucky) encourage others to add their knowledge.}

I'm involved with the FGE pages, and was planning a few PCM posts 

\textcolor{blue}{Minor Comments
Abstract: This should be in the proper BES-approved format (i.e. as a bulleted list).}

Fixed

\textcolor{blue}{l315-6: A proper citation is needed, I think. Because who wouldn't want to see this cited formally?}

Exactly how a song should be cited is not entirely clear but we have added a the following citation: Manic Street Preachers. 1996. Mr Carbohydrate. A Design for Life [CD 1], track 2. Columbia Records.


\textcolor{blue}{\textbf{Referee 1: Amy Zanne}}\\

\textcolor{blue}{As an audience member at this symposium and PCM end user, I was made well aware of the nature of use and abuse of PCMs. I walked away from this symposium oddly hopeful though that through collaborations, discussions and a number of items detailed in this review, solutions to intractable problems and public awareness of misuse can be addressed. I enjoyed this review as it provided concrete identification of the problems facing us without pointing fingers at anyone in particular. It also provided tractable solutions for how we can move forward. I found it a nice way to pause and take stock of where we are and how we can better go in the future.}

\textcolor{blue}{My biggest suggestion is that there can be some tightening of the text in a few places and a better mapping in others to the headings. I’ll detail this specifically below.}

Thanks for the helpful and encouraging comments. :)

\textcolor{blue}{L179 I’d suggest that some of this folklore also comes from within the labs developing PCM. The reason Sam Price and I proposed the R hackathon at NESCent was because each of us had knowledge about PCM issues and solutions that were lab based from the groups where we came from who were developing methods. Some of this folklore is based on good practices, others not but the point is it is being passed down outside of the public literature.}

We agree and make exactly these points on line XX: "Useful PCM folklore is often shared among developers, and among collaborating groups, but is not always shared outside of these circles."

\textcolor{blue}{L224-225 I think you can just say long and tedious or something along those lines. Why is “>20 pages” too long and what gives me a headache may be different to what gives you a “headache”.}

We have removed reference to headaches and also added the point that often equations are necessary.

\textcolor{blue}{L231 “This” is somewhat unclear. I’d make it clear that this review led to you needing to read more than 300 manuscript pages and a book to fully understand the issue.}

We have expended on this to make the point that the difficulty relates to the problem of how much reading is necessary to have sufficient understanding. 

\textcolor{blue}{L234 “required” for what? I’m assuming to correctly run the analysis or something along those lines?}

\textcolor{blue}{L257 drop a “to”}

\textcolor{blue}{L279 Again “incomprehensible” seems to depend on who is reading it. I’d argue that most methods papers are written for other methods developers as those are the audiences that will be reviewing the paper. They may or may not find the papers incomprehensible. The papers are likely to be less accessible but important for end users.}

\textcolor{blue}{L284 I’d go further than saying “there’s no harm in doing”. In fact there’s a lot of benefit (carrots). Although as you later note, we currently have no sticks making us do this. This seems like the best way to get the newly developed method into the hands of end users, make it accessible, but I agree that it is more work.}

\textcolor{blue}{L287 “help” with what? I fully agree that reproducibility is important in science but I’m unclear how it helps with this particular issue of “simplify, summarise and share”. I can imagine some avenues but I think it would be helpful to spell them out for the reader as otherwise it seems off topic.}

\textcolor{blue}{L297 “bigger problem” for what? I’m guessing to come up to speed with understanding how the method works.}

\textcolor{blue}{L305 I’ve appreciated the Prometheus wiki ($http://prometheuswiki.publish.csiro.au/tiki-custom_home.php$) as a place for methods to be posted and reviewed, as well as providing an avenue for discussion of those methods. You have a number listed here so don’t need another per se but I think this is a nice and well-used resource that would serve as a nice model for PCMs.}

\textcolor{blue}{L311 I think there’s plenty of incentive but it’s the carrot kind (it gets the PCM in an accessible form into the hands of an end user, meaning it’s more likely to be used/reused) rather than the stick. Perhaps we do better with sticks than carrots, which is why Genbank works and more people are making public their data as journals require it?}

\textcolor{blue}{L313 To me this section reads as a series of suggestions rather than that we should be more cynical. I might broaden this subject heading. For instance, beginning with “It” on L323 this seems to be about best practices and things to keep in mind rather than being cynical. Basically you are outlining a nice checklist that we should keep in mind for most analyses (i.e., what’s the sample size needed for enough power, do we need a complete and fully bifurcated phylogeny, etc.). Such a list would make a nice table… There’s also of course assumptions/issues to consider that are very tied to specific PCMs. So, my take home is there are some things to keep in mind generally across most/all analyses and there are things to keep in mind specific to this analysis you are considering (the latter is harder to come by). There are a lot of “consider” and “for example” in this section.}

\textcolor{blue}{L341 It might also be useful to mention that reviewers and editors sometimes require authors to run PCMs even when these were not part of the original plan for the project. Sometimes these are useful suggestions that help to broaden the question, other times they aren’t appropriate. It would be good to empower authors to say no when it’s the latter.}

\textcolor{blue}{L346 Again this section is more than just “Incentives” it feels perhaps like “Solutions” or something like that.}

\textcolor{blue}{L359 I agree that assumptions are not always given or are hard to find. Perhaps add “lists and” following “include”}

\textcolor{blue}{L364 I fully agree that it’s hard to find funding for method development but it’s also hard to find funding for simple data collection. I think what drives finding are theoretical/hypothesis driven science questions.}

\textcolor{blue}{L366 I don’t understand what this sentence means as written “This unfairly prejudices funding bodies (and hiring committees)”. Define “this” perhaps. I’d take it that funding bodies and hiring committees are creating these “biases” and driving them forward. We certainly need to change how the field appreciates methods developers but it comes at all of these levels.}

\textcolor{blue}{L372 “get” should be “gets”}

\textcolor{blue}{L386 Insert “is” before “becoming”}


\textcolor{blue}{Reviewer 2: Simon Blomberg}

\textcolor{blue}{After reading the abstract of this paper, I was sceptical. However, upon reading it I found that it makes some reasonable points. First I will discuss the negative aspects.}

\textcolor{blue}{My main issue is with sections 2 (beginning line 218) and 3 (beginning line 238). The fact that the literature is too difficult and opaque for some readers is not the fault of the literature. These statistical methods are NOT easy, so researchers will need to do due diligence and take the time to learn this stuff properly (including caveats and assumptions, when they are known and described in the literature). In any statistical analysis, the buck stops with the authors. It is your responsibility to get it right. You can't complain that you are using methods that you don't understand. If that is the case you should go back and learn more about the methods. Complaining that there is too much maths and it gives you a headache (lines 224-225) is not a reason to castigate the authors of methods. Mathematics is the language of modern evolutionary biology and the sooner people get used to that, the better. Don't like maths? Don't do quantitative science. Complaining that the literature is too big and difficult and going ahead with an analysis anyway is just academic laziness. Don't want to get to grips with the literature? Collaborate with someone who knows it, like a statistician. The authors appear to be promoting a new section for such papers, as assumptions and caveats can be in the Intro, Methods, Results and/or Discussion (lines 235-237). Well how about this: read the papers properly! Again, laziness here.}

We have removed reference to headaches and revised these sections substantially. We agree that due diligence is key. However our point is that due diligence is not just the responsibility of readers/users but also of developers. It is not simply laziness on the readers side that leads to misunderstanding. Often papers, especially those with software, are presented as if the method is faultless. This is a symptom of how papers are published but can very easily mislead novice users. Clarity in writing is the responsibility of the write, thorough reading is the responsibility of the reader.

\textcolor{blue}{The authors complain that researchers often jump straight into using a method (line 259-263) without knowing its limitations. Well this is just a problem with due diligence again. It's not the method developer's fault. The authors complain that this seems to be particularly a problem with methods implemented in R. Well, R is free software. You have the source code. If you really want to know how a method works, study the source code! You can't do that for other, proprietary software. You have to trust the company/lab that produces the software that it does what is claimed on the box. Using free software like R is thus more scientific than using proprietary software. It \_allows\_ researchers to do due diligence.}

We agree. However, as we note above we also argue that doing due diligence is a responsibility of both the users (in reading widely and deeply) and the developers (in writing clearly and stating assumptions, limitations etc). 

\textcolor{blue}{The authors claim that novice users may fail to understand a method because the accompanying paper is incomprehensible (lines 277-279). Well, that novice needs to do more work. Usually that involves reading around the topic, reading the source code, and maybe running some simulations. Students should be doing this all the time: it's called \_learning\_. It's their job. In fact it is every researcher's job.}

We absolutely agree that readers must do the work to understand methods but reiterate the point above - due diligence is a responsibility for developers as well as users. We have revised lines 277-279 accordingly.

\textcolor{blue}{And now for the positive points.}

\textcolor{blue}{The authors ask for more and better documentation, including YouTube tutorials, blog posts, etc. This is to be commended. We always need more, better documentation. Like scientists who always need more, better data. Unfortunately, most phylogenetic comparative methods are developed by academic researchers who may not have the resources (time or money) to do all this. That they offer their work for free (in R) is a major gift to the scientific community. I have a suggestion: Don't like the (lack of) documentation? Write it yourself! You will learn way more than just blindly going ahead and using a method uncritically. If it's an R package, contact the author and say you would like to contribute to the documentation. I suspect that most R developers would jump at the chance to have someone provide more/better documentation. This could be directly editing the package help files and/or providing vignettes. You  might even be included as an author of the package. If you really have a lot to say about methods, write a book! There are plenty of books on R, including a couple that describe PCMs. Again, more/better books are welcome. Another solution would be to start giving workshops and seminars on a fee for service basis. The fees could be ploughed back into improving the software and documentation. Workshops on statistical analysis (of any kind) are very popular and you do not need to be a package's developer to deliver these. See  http://www.highstat.com/ for an example of how to make a business out of it. I'm not sure that the evolutionary biology community is big enough to support this, but it may be.}

\textcolor{blue}{The authors encourage more reproducible research and open source science (lines 287-290). This is a growing movement and will no doubt help raise the quality of research.}

\textcolor{blue}{The authors discuss editorial policies allowing incremental improvements and diagnostics for PCMs. This would also be of value. I also applaud the incentives suggested in lines 346-378. They would be most welcome.}

\textcolor{blue}{Minor criticisms:}

\textcolor{blue}{line 124-125. "Brownian constant variance model" is a complete contradiction in terms. Brownian motion has infinite total variation and quadratic variation increasing with time. If a process has constant variance, it's not Brownian motion!}

This was an unfortunate error on our part. Of course the variance is not constant in the Brownian motion model. We have replaced the two instances of "constant variance" with "motion".

\textcolor{blue}{line 145-150. It may be that a particular data set cannot support more than one tree-wise set of parameters. But the authors are right. In general there is no reason to expect that a whole tree would be governed by just one process. Of course, you can fit more complicated models in different parts of the tree, but again the quality of the data may not make this feasible.}

\textcolor{blue}{line 339-340: It's just not true that, "non-ultrametric trees cause variance-covariance matrices to become non-tree like". You can jump back and forth between a tree representation and a matrix representation, and it is just not affected by ultrametricity. In the ape package, there is vcv.phylo to convert a tree to a matrix, and vcv2phylo to convert a matrix back to a tree. (I wrote the latter.)}

We believe that this is a misunderstanding of our point and reflects a lack of clarity in our writing. We now write: 

"Some current implementations of the OU model should not be used with non-ultrametric trees (e.g. MOTMOT; \citealp{Thomas:2011aa}) because they are based on transforming the tree directly, rather than transforming the variance co-variance matrix. The problem is that where at least one tip does not survive to the present the expected covariances relating each of those two tips with any other tip in the tree are not identical. Worked examples and explanations are provided in \citep{slater2014correction}. Although this is not a problem with applying the OU model to non-ultrametric trees per se, it is an example of different implementations of a common model that some users may not be aware of. "

Let us know if you require any further information,

\closing{Yours sincerely,}

\end{letter}
\end{document}